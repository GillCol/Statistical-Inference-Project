\PassOptionsToPackage{unicode=true}{hyperref} % options for packages loaded elsewhere
\PassOptionsToPackage{hyphens}{url}
%
\documentclass[
]{article}
\usepackage{lmodern}
\usepackage{amssymb,amsmath}
\usepackage{ifxetex,ifluatex}
\ifnum 0\ifxetex 1\fi\ifluatex 1\fi=0 % if pdftex
  \usepackage[T1]{fontenc}
  \usepackage[utf8]{inputenc}
  \usepackage{textcomp} % provides euro and other symbols
\else % if luatex or xelatex
  \usepackage{unicode-math}
  \defaultfontfeatures{Scale=MatchLowercase}
  \defaultfontfeatures[\rmfamily]{Ligatures=TeX,Scale=1}
\fi
% use upquote if available, for straight quotes in verbatim environments
\IfFileExists{upquote.sty}{\usepackage{upquote}}{}
\IfFileExists{microtype.sty}{% use microtype if available
  \usepackage[]{microtype}
  \UseMicrotypeSet[protrusion]{basicmath} % disable protrusion for tt fonts
}{}
\makeatletter
\@ifundefined{KOMAClassName}{% if non-KOMA class
  \IfFileExists{parskip.sty}{%
    \usepackage{parskip}
  }{% else
    \setlength{\parindent}{0pt}
    \setlength{\parskip}{6pt plus 2pt minus 1pt}}
}{% if KOMA class
  \KOMAoptions{parskip=half}}
\makeatother
\usepackage{xcolor}
\IfFileExists{xurl.sty}{\usepackage{xurl}}{} % add URL line breaks if available
\IfFileExists{bookmark.sty}{\usepackage{bookmark}}{\usepackage{hyperref}}
\hypersetup{
  pdftitle={Coursera Statistical Inference Course Project part 1},
  pdfauthor={Gill Collier},
  pdfborder={0 0 0},
  breaklinks=true}
\urlstyle{same}  % don't use monospace font for urls
\usepackage[margin=1in]{geometry}
\usepackage{color}
\usepackage{fancyvrb}
\newcommand{\VerbBar}{|}
\newcommand{\VERB}{\Verb[commandchars=\\\{\}]}
\DefineVerbatimEnvironment{Highlighting}{Verbatim}{commandchars=\\\{\}}
% Add ',fontsize=\small' for more characters per line
\usepackage{framed}
\definecolor{shadecolor}{RGB}{248,248,248}
\newenvironment{Shaded}{\begin{snugshade}}{\end{snugshade}}
\newcommand{\AlertTok}[1]{\textcolor[rgb]{0.94,0.16,0.16}{#1}}
\newcommand{\AnnotationTok}[1]{\textcolor[rgb]{0.56,0.35,0.01}{\textbf{\textit{#1}}}}
\newcommand{\AttributeTok}[1]{\textcolor[rgb]{0.77,0.63,0.00}{#1}}
\newcommand{\BaseNTok}[1]{\textcolor[rgb]{0.00,0.00,0.81}{#1}}
\newcommand{\BuiltInTok}[1]{#1}
\newcommand{\CharTok}[1]{\textcolor[rgb]{0.31,0.60,0.02}{#1}}
\newcommand{\CommentTok}[1]{\textcolor[rgb]{0.56,0.35,0.01}{\textit{#1}}}
\newcommand{\CommentVarTok}[1]{\textcolor[rgb]{0.56,0.35,0.01}{\textbf{\textit{#1}}}}
\newcommand{\ConstantTok}[1]{\textcolor[rgb]{0.00,0.00,0.00}{#1}}
\newcommand{\ControlFlowTok}[1]{\textcolor[rgb]{0.13,0.29,0.53}{\textbf{#1}}}
\newcommand{\DataTypeTok}[1]{\textcolor[rgb]{0.13,0.29,0.53}{#1}}
\newcommand{\DecValTok}[1]{\textcolor[rgb]{0.00,0.00,0.81}{#1}}
\newcommand{\DocumentationTok}[1]{\textcolor[rgb]{0.56,0.35,0.01}{\textbf{\textit{#1}}}}
\newcommand{\ErrorTok}[1]{\textcolor[rgb]{0.64,0.00,0.00}{\textbf{#1}}}
\newcommand{\ExtensionTok}[1]{#1}
\newcommand{\FloatTok}[1]{\textcolor[rgb]{0.00,0.00,0.81}{#1}}
\newcommand{\FunctionTok}[1]{\textcolor[rgb]{0.00,0.00,0.00}{#1}}
\newcommand{\ImportTok}[1]{#1}
\newcommand{\InformationTok}[1]{\textcolor[rgb]{0.56,0.35,0.01}{\textbf{\textit{#1}}}}
\newcommand{\KeywordTok}[1]{\textcolor[rgb]{0.13,0.29,0.53}{\textbf{#1}}}
\newcommand{\NormalTok}[1]{#1}
\newcommand{\OperatorTok}[1]{\textcolor[rgb]{0.81,0.36,0.00}{\textbf{#1}}}
\newcommand{\OtherTok}[1]{\textcolor[rgb]{0.56,0.35,0.01}{#1}}
\newcommand{\PreprocessorTok}[1]{\textcolor[rgb]{0.56,0.35,0.01}{\textit{#1}}}
\newcommand{\RegionMarkerTok}[1]{#1}
\newcommand{\SpecialCharTok}[1]{\textcolor[rgb]{0.00,0.00,0.00}{#1}}
\newcommand{\SpecialStringTok}[1]{\textcolor[rgb]{0.31,0.60,0.02}{#1}}
\newcommand{\StringTok}[1]{\textcolor[rgb]{0.31,0.60,0.02}{#1}}
\newcommand{\VariableTok}[1]{\textcolor[rgb]{0.00,0.00,0.00}{#1}}
\newcommand{\VerbatimStringTok}[1]{\textcolor[rgb]{0.31,0.60,0.02}{#1}}
\newcommand{\WarningTok}[1]{\textcolor[rgb]{0.56,0.35,0.01}{\textbf{\textit{#1}}}}
\usepackage{graphicx,grffile}
\makeatletter
\def\maxwidth{\ifdim\Gin@nat@width>\linewidth\linewidth\else\Gin@nat@width\fi}
\def\maxheight{\ifdim\Gin@nat@height>\textheight\textheight\else\Gin@nat@height\fi}
\makeatother
% Scale images if necessary, so that they will not overflow the page
% margins by default, and it is still possible to overwrite the defaults
% using explicit options in \includegraphics[width, height, ...]{}
\setkeys{Gin}{width=\maxwidth,height=\maxheight,keepaspectratio}
\setlength{\emergencystretch}{3em}  % prevent overfull lines
\providecommand{\tightlist}{%
  \setlength{\itemsep}{0pt}\setlength{\parskip}{0pt}}
\setcounter{secnumdepth}{-2}
% Redefines (sub)paragraphs to behave more like sections
\ifx\paragraph\undefined\else
  \let\oldparagraph\paragraph
  \renewcommand{\paragraph}[1]{\oldparagraph{#1}\mbox{}}
\fi
\ifx\subparagraph\undefined\else
  \let\oldsubparagraph\subparagraph
  \renewcommand{\subparagraph}[1]{\oldsubparagraph{#1}\mbox{}}
\fi

% set default figure placement to htbp
\makeatletter
\def\fps@figure{htbp}
\makeatother


\title{Coursera Statistical Inference Course Project part 1}
\author{Gill Collier}
\date{06/11/2020}

\begin{document}
\maketitle

\hypertarget{gill-collier---01-november-2020}{%
\section{Gill Collier - 01 November
2020}\label{gill-collier---01-november-2020}}

\hypertarget{coursera-statistical-inference---course-project}{%
\section{Coursera: Statistical Inference - Course
Project}\label{coursera-statistical-inference---course-project}}

\hypertarget{part-1-simulation-exercise-instructions}{%
\section{Part 1: Simulation Exercise
Instructions}\label{part-1-simulation-exercise-instructions}}

\hypertarget{in-this-project-you-will-investigate-the-exponential-distribution-in-r-and}{%
\section{In this project you will investigate the exponential
distribution in R
and}\label{in-this-project-you-will-investigate-the-exponential-distribution-in-r-and}}

\hypertarget{compare-it-with-the-central-limit-theorem.-the-exponential-distribution-can}{%
\section{compare it with the Central Limit Theorem. The exponential
distribution
can}\label{compare-it-with-the-central-limit-theorem.-the-exponential-distribution-can}}

\hypertarget{be-simulated-in-r-with-rexpn-lambda-where-lambda-is-the-rate-parameter.}{%
\section{be simulated in R with rexp(n, lambda) where lambda is the rate
parameter.}\label{be-simulated-in-r-with-rexpn-lambda-where-lambda-is-the-rate-parameter.}}

\hypertarget{the-mean-of-exponential-distribution-is-1lambda-and-the-standard-deviation}{%
\section{The mean of exponential distribution is 1/lambda and the
standard
deviation}\label{the-mean-of-exponential-distribution-is-1lambda-and-the-standard-deviation}}

\hypertarget{is-also-1lambda.-set-lambda-0.2-for-all-of-the-simulations.-you-will}{%
\section{is also 1/lambda. Set lambda = 0.2 for all of the simulations.
You
will}\label{is-also-1lambda.-set-lambda-0.2-for-all-of-the-simulations.-you-will}}

\hypertarget{investigate-the-distribution-of-averages-of-40-exponentials.-note-that-you}{%
\section{investigate the distribution of averages of 40 exponentials.
Note that
you}\label{investigate-the-distribution-of-averages-of-40-exponentials.-note-that-you}}

\hypertarget{will-need-to-do-a-thousand-simulations.}{%
\section{will need to do a thousand
simulations.}\label{will-need-to-do-a-thousand-simulations.}}

\hypertarget{illustrate-via-simulation-and-associated-explanatory-text-the-properties-of}{%
\section{Illustrate via simulation and associated explanatory text the
properties
of}\label{illustrate-via-simulation-and-associated-explanatory-text-the-properties-of}}

\hypertarget{the-distribution-of-the-mean-of-40-exponentials.-you-should}{%
\section{the distribution of the mean of 40 exponentials. You
should:}\label{the-distribution-of-the-mean-of-40-exponentials.-you-should}}

\hypertarget{show-the-sample-mean-and-compare-it-to-the-theoretical-mean-of-the}{%
\section{1. Show the sample mean and compare it to the theoretical mean
of
the}\label{show-the-sample-mean-and-compare-it-to-the-theoretical-mean-of-the}}

\hypertarget{distribution.}{%
\section{distribution.}\label{distribution.}}

\hypertarget{show-how-variable-the-sample-is-via-variance-and-compare-it-to-the}{%
\section{2. Show how variable the sample is (via variance) and compare
it to
the}\label{show-how-variable-the-sample-is-via-variance-and-compare-it-to-the}}

\hypertarget{theoretical-variance-of-the-distribution.}{%
\section{theoretical variance of the
distribution.}\label{theoretical-variance-of-the-distribution.}}

\hypertarget{show-that-the-distribution-is-approximately-normal.}{%
\section{3. Show that the distribution is approximately
normal.}\label{show-that-the-distribution-is-approximately-normal.}}

\hypertarget{in-point-3-focus-on-the-difference-between-the-distribution-of-a-large}{%
\section{In point 3, focus on the difference between the distribution of
a
large}\label{in-point-3-focus-on-the-difference-between-the-distribution-of-a-large}}

\hypertarget{collection-of-random-exponentials-and-the-distribution-of-a-large-collection}{%
\section{collection of random exponentials and the distribution of a
large
collection}\label{collection-of-random-exponentials-and-the-distribution-of-a-large-collection}}

\hypertarget{of-averages-of-40-exponentials.}{%
\section{of averages of 40
exponentials.}\label{of-averages-of-40-exponentials.}}

\hypertarget{libraries}{%
\section{Libraries}\label{libraries}}

\begin{Shaded}
\begin{Highlighting}[]
\KeywordTok{library}\NormalTok{(ggplot2)}
\end{Highlighting}
\end{Shaded}

\begin{verbatim}
## Warning: replacing previous import 'vctrs::data_frame' by 'tibble::data_frame'
## when loading 'dplyr'
\end{verbatim}

\hypertarget{set-variables}{%
\section{Set variables}\label{set-variables}}

\begin{Shaded}
\begin{Highlighting}[]
\KeywordTok{set.seed}\NormalTok{(}\DecValTok{123}\NormalTok{)}
\NormalTok{lambda <-}\StringTok{ }\FloatTok{0.2}
\NormalTok{n <-}\StringTok{ }\DecValTok{40}
\end{Highlighting}
\end{Shaded}

\hypertarget{generate-the-sample-means}{%
\section{Generate the sample means}\label{generate-the-sample-means}}

\begin{Shaded}
\begin{Highlighting}[]
\NormalTok{mns =}\StringTok{ }\OtherTok{NULL}
\ControlFlowTok{for}\NormalTok{ (i }\ControlFlowTok{in} \DecValTok{1} \OperatorTok{:}\StringTok{ }\DecValTok{1500}\NormalTok{) mns =}\StringTok{ }\KeywordTok{c}\NormalTok{(mns, }\KeywordTok{mean}\NormalTok{(}\KeywordTok{rexp}\NormalTok{(n, lambda)))}
\KeywordTok{summary}\NormalTok{(mns)}
\end{Highlighting}
\end{Shaded}

\begin{verbatim}
##    Min. 1st Qu.  Median    Mean 3rd Qu.    Max. 
##   2.943   4.465   4.964   4.996   5.517   7.717
\end{verbatim}

\hypertarget{calculate-the-mean-of-these-means}{%
\section{Calculate the mean of these
means}\label{calculate-the-mean-of-these-means}}

\begin{Shaded}
\begin{Highlighting}[]
\NormalTok{m_mns <-}\StringTok{ }\KeywordTok{mean}\NormalTok{(mns)}
\end{Highlighting}
\end{Shaded}

\hypertarget{calculate-the-theoretical-mean}{%
\section{Calculate the theoretical
mean}\label{calculate-the-theoretical-mean}}

\begin{Shaded}
\begin{Highlighting}[]
\NormalTok{t_mns <-}\StringTok{ }\NormalTok{lambda}\OperatorTok{^-}\DecValTok{1}
\NormalTok{t_mns}
\end{Highlighting}
\end{Shaded}

\begin{verbatim}
## [1] 5
\end{verbatim}

\begin{Shaded}
\begin{Highlighting}[]
\KeywordTok{abs}\NormalTok{(m_mns }\OperatorTok{-}\StringTok{ }\NormalTok{t_mns)}
\end{Highlighting}
\end{Shaded}

\begin{verbatim}
## [1] 0.003612815
\end{verbatim}

\hypertarget{the-central-limit-theorem-states-that-the-sampling-distribution-of-a-sample}{%
\section{The central limit theorem states that the sampling distribution
of a
sample}\label{the-central-limit-theorem-states-that-the-sampling-distribution-of-a-sample}}

\hypertarget{mean-is-approximately-normal-if-the-sample-size-is-large-enough-even-if}{%
\section{mean is approximately normal if the sample size is large
enough, even
if}\label{mean-is-approximately-normal-if-the-sample-size-is-large-enough-even-if}}

\hypertarget{the-population-distribution-is-not-normal.}{%
\section{the population distribution is not
normal.}\label{the-population-distribution-is-not-normal.}}

\hypertarget{the-clt-in-this-simulation-appears-to-be-valid-as-this-shows-that-increasing}{%
\section{The CLT in this simulation appears to be valid as this shows
that
increasing}\label{the-clt-in-this-simulation-appears-to-be-valid-as-this-shows-that-increasing}}

\hypertarget{the-number-of-samples-narrows-the-gap-between-the-simulation-mean-and-the}{%
\section{the number of samples narrows the gap between the simulation
mean and
the}\label{the-number-of-samples-narrows-the-gap-between-the-simulation-mean-and-the}}

\hypertarget{theoretical-mean}{%
\section{theoretical mean}\label{theoretical-mean}}

\hypertarget{calculate-the-sample-variance}{%
\section{Calculate the sample
variance}\label{calculate-the-sample-variance}}

\begin{Shaded}
\begin{Highlighting}[]
\NormalTok{s_var <-}\StringTok{ }\KeywordTok{var}\NormalTok{(mns)}
\NormalTok{s_var}
\end{Highlighting}
\end{Shaded}

\begin{verbatim}
## [1] 0.5952215
\end{verbatim}

\hypertarget{calculate-the-theoretical-variance}{%
\section{Calculate the theoretical
variance}\label{calculate-the-theoretical-variance}}

\begin{Shaded}
\begin{Highlighting}[]
\NormalTok{t_var <-}\StringTok{ }\NormalTok{(lambda }\OperatorTok{*}\StringTok{ }\KeywordTok{sqrt}\NormalTok{(n)) }\OperatorTok{^}\StringTok{ }\DecValTok{-2}
\NormalTok{t_var}
\end{Highlighting}
\end{Shaded}

\begin{verbatim}
## [1] 0.625
\end{verbatim}

\hypertarget{compare-the-sample-variance-to-the-theoretical-variance}{%
\section{Compare the sample variance to the theoretical
variance}\label{compare-the-sample-variance-to-the-theoretical-variance}}

\begin{Shaded}
\begin{Highlighting}[]
\NormalTok{s_var }\OperatorTok{-}\StringTok{ }\NormalTok{t_var}
\end{Highlighting}
\end{Shaded}

\begin{verbatim}
## [1] -0.02977846
\end{verbatim}

\hypertarget{this-comparison-shows-there-is-only-a-small-difference-between-the}{%
\section{This comparison shows there is only a small difference between
the}\label{this-comparison-shows-there-is-only-a-small-difference-between-the}}

\hypertarget{simulation-variance-and-the-theorectical-variance}{%
\section{simulation variance and the theorectical
variance}\label{simulation-variance-and-the-theorectical-variance}}

\hypertarget{the-distribution-can-be-shown-in-a-histogram.-this-shows-the-sample-means}{%
\section{The distribution can be shown in a histogram. This shows the
sample
means}\label{the-distribution-can-be-shown-in-a-histogram.-this-shows-the-sample-means}}

\hypertarget{from-the-simulation-with-an-overlay-of-a-normal-distribution.}{%
\section{from the simulation with an overlay of a normal
distribution.}\label{from-the-simulation-with-an-overlay-of-a-normal-distribution.}}

\begin{Shaded}
\begin{Highlighting}[]
\KeywordTok{hist}\NormalTok{(mns, }\DataTypeTok{prob =} \OtherTok{TRUE}\NormalTok{, }\DataTypeTok{col =} \StringTok{"light blue"}\NormalTok{, }\DataTypeTok{main =} 
             \StringTok{"Histogram of Means from Simulation"}\NormalTok{, }\DataTypeTok{xlab =} \StringTok{"simulation mean"}\NormalTok{, }
     \DataTypeTok{breaks =} \DecValTok{20}\NormalTok{)}
\end{Highlighting}
\end{Shaded}

\includegraphics{Coursera-Statistical-Inference-Course-Project-part-1_files/figure-latex/unnamed-chunk-9-1.pdf}

\end{document}
